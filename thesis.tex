
\documentclass[a4paper,singleside,12pt]{report} % Uncomment this for single side pdf.
%\documentclass[a4paper,twoside,12pt]{report} % Uncomment this for printing.

\usepackage{ai_bo_thesis}
\usepackage[english]{babel}
\usepackage[T1]{fontenc}
\usepackage{lmodern}


\usepackage[backend=biber,style=trad-plain, sorting=nty,firstinits=true]{biblatex}
\addbibresource{biblio.bib}

\setmainfont{Times New Roman}
\begin{document}
	
	\title{Towards sequential problem solving in ACT-R: a case study of TANGRAM}
	\topic{Cognitive modeling for robotics in manufacturing}
	\candidate{Giacomo Zamprogno}
	\supervisor{Prof. Giuseppe Di Pellegrino}
	\cosupervisor{co-supervisor} % One co-supervisor.
	%\cosupervisors{} % More than one co-supervisor.
	\academicyear{2021-2022}
	\session{1st}
	
	\frontispiece 
	\dedication{+retrieval> \\ ISA dedication \\ NAME tbd \\ REASON tbd}
	\toc
	\figstoc
	\tablestoc
	\begintext
	
	\chapter{Introduction}
    Where I describe cogntive modelling and its applications, and the specific case study
    of the tangram in the context manufacturing and tutoring
    \\
    Cognitive architectures and AI \cite{Lieto20181}
    
    \chapter{Related Works}
    \textit{Where I quickly go throught the available Literature, describe ACT-R and its functioning and analyze the 
	various approaches for tangram solving}
	\subsection{ACT-R}
	\subsection{Cogntive modelling of puzzles}
	Despite their nature and potential as an abstraction for more sophisticated sequential problem solving tasks,
	the applicaions of cognitive modelling to puzzles are still at an early stage.\\
	Rosenberg et al. \cite{social-robot} coupled cognitive architectures and the tangram puzzle	in order to model 
	the curiosity aspect of a social robot, but the actual solution of the puzzle was implemented with a connectionist approach
	and the cognitive aspect was focused on the social interaction and artificial curioisy modelling.
	Gentile and Lieto \cite{GENTILE20221} instead used ACT-R in order to model the role of mental rotation applied at the task of
	the Tetris\textsuperscript{TM} video-game, based on the previous work of Shepard and Metzler\cite{shepard1971mental}, providing
	introductory results and a functioning model for the mental rotation process.
	\subsection{The Tangram}
    \chapter{Experimental scenario}
    Where I describe the performed experiments
    \chapter{Data Analysis and Hypothesis}
    Where I qualitatively and quantitatively analyze the available data, provide figures and introduce
    the leading Hypothesis that the model will try to explain
    \chapter{Model description}
    Where I provide a detailed description of the model and the modelling choices
    \chapter{Results and discussion}
    Where I compare the model to the expected data and try to discuss whether the hypothesis are funded
    and whether there are possible applications
	\appendix
	
	\printbibliography[heading=bibintoc] % biblatex

	
	\acknowledgements
		
		
\end{document}