
\documentclass[a4paper,singleside,12pt]{report} % Uncomment this for single side pdf.
%\documentclass[a4paper,twoside,12pt]{report} % Uncomment this for printing.

\usepackage{ai_bo_thesis}
\usepackage[english]{babel}
\usepackage[T1]{fontenc}
\usepackage{lmodern}


\usepackage[backend=biber,style=trad-plain, sorting=nty,firstinits=true]{biblatex}
\addbibresource{biblio.bib}

\setmainfont{Times New Roman}
\begin{document}
	
	\title{Towards sequential problem solving in ACT-R: a case study of TANGRAM}
	\topic{Cognitive modeling for robotics in manufacturing}
	\candidate{Giacomo Zamprogno}
	\supervisor{Prof. Giuseppe Di Pellegrino}
	\cosupervisor{co-supervisor} % One co-supervisor.
	%\cosupervisors{} % More than one co-supervisor.
	\academicyear{2021-2022}
	\session{1st}
	
	\frontispiece 
	\dedication{+retrieval> \\ ISA dedication \\ NAME tbd \\ REASON tbd}
	\toc
	\figstoc
	\tablestoc
	\begintext
	
	\chapter{Introduction}
    \textit{Where I describe cognitive modelling and its applications, and the specific case study
    of the tangram in the context manufacturing and tutoring}
	
	What is cognitive modelling? Why how does it relate to AI and why is it of interest at airbus?
	MAYBE the reserch group at airbus
	The idea of the project
	\section{The Tangram}
	The tangram is an ancient Chinese puzzle in which seven pieces, also called \textit{tans} are obtained
	from an original square.\\
	The most common tans, also used in this work, consist of 5 square triangles (2 small, 1 medium
	and 2 large), 1 square and 1 parallelogram, their dimension is shown in [reference figure].\\
	Usually players are presented with an homogeneous silhouette, likely in the shape of some stylized figure, and are tasked 
	with reproducing such pattern by using all the tans, without overlap.\\
	Besides the interest among puzzle-solvers, a number of works has been studying its applications in the teaching fields,
	where its nature as a fairly complex game can help children to develop geometric and communicative skills[citation to some papers
	might be needed here].\\
	In the context of cognitive modelling, the tangram can be seen as an abstraction for a set of different sequential problem solving tasks.
	The fact that the field is still at the early stages of development, studying a simpler puzzle and how humans approach its solution might provide 
	initial insights about various types of assembling processes, in an attempt to create machines more capable to interpret and adapt to human 
	actions in an explainable and rule-based way. 	
    
    
    \chapter{Related Works}
    \textit{Where I quickly go throught the available Literature, describe ACT-R and its functioning and analyze the 
	various approaches for tangram solving}
	\section{ACT-R}
	\section{Cogntive modelling of puzzles}
	Despite their nature and potential as an abstraction for more sophisticated sequential problem solving tasks,
	the applicaions of cognitive modelling to puzzles are still at an early stage.\\
	Rosenberg et al. \cite{social-robot} coupled cognitive architectures and the tangram puzzle	in order to model 
	the curiosity aspect of a social robot, but the actual solution of the puzzle was implemented with a connectionist approach
	and the cognitive aspect was focused on the social interaction and artificial curioisy modelling.
	Gentile and Lieto \cite{GENTILE20221} instead used ACT-R in order to model the role of mental rotation applied at the task of
	the Tetris\textsuperscript{TM} video-game, based on the previous work of Shepard and Metzler\cite{shepard1971mental}, providing
	introductory results and a functioning model for the mental rotation process.
	\section{The Tangram}
	As mentioned, cognitive-modelling specific literature regarding the tangram is extremely limited. Nonetheless, previous works in the field
	of computer science and cognitive psychology can be seen as an interesting starting baseline for the modelling process. 
	\\Among the computational approaches, Deutsch and Hayes \cite{heuristic} originally suggested a heuristic solution based on extending the lines inside the silhouette and
	matching the edges of the tans with the prolonged lines; while their \textit{direct-match} and \textit{2.5-3.5} rules might resemble what will
	be later defined as \textit{perfect fit}, their fully geometrical algorithm, and especially the evaluations on the extendeded lines are unlikely 
	to have cognitive plausibility.\\
	During the advent of machine learning, Oflazer \cite{Oflazer1993SolvingTP} used a Boltzmann Machine, where each neuron represented a possible positioning of one piece, and was fed 
	with excitatory input from the outline constraints and inhibitory input from the conflicting configurations. The main limit of the approach is that the authors only tackle \textit{grid tangrams}, 
	where each corner of a placed tan must be coincident with a point in an equally spaced grid. This in turn largely limits the allowed rotations for the pieces and thus its applicability,
	even at a knwoledge represenation level, when dealing with less constrained puzzles as the irrational length of edges would be impossible to tackle using the grid. \\


    
	
	\chapter{Experimental scenario}
    In collaboration with the Technical University of Berlin [shall I actually mention it?] an experiment was developed and performed in order to obtain the training and test sets.\\
	Each participant was presented with a virtual implmentation of the tanrgam game, including 4 puzzles [cite figure]. After an inital explanation of the controls, each participant was required tackle
	the solution of the puzzles, which were always shwon in the same order. There was no time limit, and at any moment the \textit{NEXT} button was available, so that the player could give up on the specific
	tangram and move to the next one. It must be noted that while backtracking was possible during the solution of the single task, once the button was pressed the solution was submitted and no option to go back
	was available.\\
	Between each problem, the participants were asked to complete a NASA evaluation form, and at the end of the trial they provided a general feedback. \\
	In addition to this, the screen recording and application logs, including data regarding times, piece action types (rotation, movement) and piece positioning were stored for the analysis. 
	During the first trial set, 31 participants (17 males, 14 females) took part to the experiment; they all had an academic background, a large majority of them being from engineering or human factors studies. \\
	These first players composed the training set.
    
	
	\chapter{Data Analysis and Hypothesis}
    Where I qualitatively and quantitatively analyze the available data, provide figures and introduce
    the leading Hypothesis that the model will try to explain
    
	
	\chapter{Model description}
    Where I provide a detailed description of the model and the modelling choices
    
	
	\chapter{Results and discussion}
    Where I compare the model to the expected data and try to discuss whether the hypothesis are funded
    and whether there are possible applications
	
	
	\appendix
	
	\printbibliography[heading=bibintoc] % biblatex

	
	\acknowledgements
		
		
\end{document}